\documentclass[a4paper,11pt]{article}
\usepackage{amssymb}
\usepackage[polish]{babel}
\usepackage[utf8]{inputenc}
\usepackage[T1]{fontenc}
\usepackage{graphicx}
\usepackage{anysize}
\usepackage{enumerate}
\usepackage{times}
\usepackage{geometry}
\usepackage{amsthm}
\usepackage{pgfplots}

\usepackage[intlimits]{amsmath}
\marginsize{3cm}{3cm}{1.5cm}{1.5cm}
\sloppy

\begin{document}
\begin{table}[ht]
\centering
\hspace*{-1cm}
\begin{tabular}{lllllll}
\cline{1-6}
\multicolumn{1}{|c|}{\begin{tabular}[c]{@{}c@{}}EAIiIB\\ Informatyka\end{tabular}}              & \multicolumn{2}{l|}{\begin{tabular}[c]{@{}l@{}}Aleksander Lisiecki\\ Natalia Materek\end{tabular}}                                                                                                & \multicolumn{1}{c|}{\begin{tabular}[c]{@{}c@{}}Rok\\ II\end{tabular}}          & \multicolumn{1}{c|}{\begin{tabular}[c]{@{}c@{}}Grupa\\ 2\end{tabular}}            & \multicolumn{1}{c|}{\begin{tabular}[c]{@{}c@{}}Zespół\\ 6\end{tabular}}      &  \\ \cline{1-6}
\multicolumn{1}{|c|}{\begin{tabular}[c]{@{}c@{}}Pracownia\\ FIZYCZNA\\ WFiIS AGH\end{tabular}} & \multicolumn{4}{l|}{\begin{tabular}[c]{@{}l@{}}Temat:\\ \textbf{\textit{Fale podłużne w ciałach stałych}} \end{tabular}}                                                                                                                                                                                                                                            & \multicolumn{1}{c|}{\begin{tabular}[c]{@{}c@{}}Nr ćwiczenia:\\ 29\end{tabular}} &  \\ \cline{1-6}
\multicolumn{1}{|l|}{\begin{tabular}[c]{@{}c@{}}Data wykonania:\\ 04.01.2017\end{tabular}}      & \multicolumn{1}{c|}{\begin{tabular}[c]{@{}c@{}}Data oddania:\\ 11.01.2017\end{tabular}} & \multicolumn{1}{l|}{\begin{tabular}[c]{@{}l@{}}Zwrot do poprawki:\\ \phantom{data poprawki}\end{tabular}} & \multicolumn{1}{l|}{\begin{tabular}[c]{@{}l@{}}Data oddania:\\  \phantom{data oddania}\end{tabular}} & \multicolumn{1}{l|}{\begin{tabular}[c]{@{}l@{}}Data zaliczenia:\\  \phantom{data zaliczenia}\end{tabular}} & \multicolumn{1}{l|}{\begin{tabular}[c]{@{}l@{}}OCENA:\\ \phantom{ocena}\end{tabular}}       &  \\ \cline{1-6}
                                                                                               &                                                                                         &                                                                                     &                                                                                &                                                                                   &                                                                               & 
\end{tabular}
\end{table}

\begin{center}
\begin{LARGE}
\textbf{Ćwiczenie nr 29: Fale podłużne w ciałach stałych}
\end{LARGE}
\end{center}

\section{Cel ćwiczenia}
\indent Wyznaczenie modułu Younga dla różnych materiałów na podstawie pomiaru prędkości rozchodzenia się fali
dźwiękowej w pręcie.

\section{Wstęp teoretyczny} 
 \indent Fala podłużna  to fala, w której drgania odbywają się w kierunku zgodnym z kierunkiem jej rozchodzenia się. 
 Opisuje ją równanie:
\begin{equation}
 y= Acos(\omega t \pm kx) 
\end{equation}  
 gdzie 
 \begin{description}
 \item[$A$] amplituda drgań [$m$]
 \item [$\omega$] prędkość kątowa $\left[\frac{1}{s}\right]$
 \item [$t$] czas [$s$]
 \item [$k$] współczynnik sprężystości [jednostka]
 \end{description}
 
\subsection{Prawa Hooke'a}
Prawo: odkształcenie jest wprost proporcjonalne do wywołującej je siły.
\begin{equation}
  \Delta l = \frac{Fl}{ES} 
\end{equation}
gdzie
\begin{description}
\item [ $\Delta l$] zmiana długości pręta [$m$]
\item [$F$] siła odkształcająca [$N$]
\item [$l$]  długość pręta [$m$]
\item [$S$]  pole przekroju pręta $\left[m^2\right]$
\item [$E$] moduł Younga $\left[\frac{N}{m^2}\right]$
\end{description}
 
 \subsection{Moduł Younga}

\indent Wychodząc od ogólnego wzoru na prawo Hooke'a: 
 \begin{equation}
 \sigma = \varepsilon E 
 \end{equation}
 gdzie
 \begin{description}
  \item [$\sigma $]  naprężenie $\left[\frac{N}{m^2}\right]$
  \item [$\varepsilon$]  odkształcenie względne [bezwymiarowe]
 \end{description}
 
 \begin{equation}
  \varepsilon = \frac{\delta \Psi}{\delta x}
 \end{equation}

\indent Otrzymujemy wzór na  prędkość rozchodzenia się fali w pręcie: 
\begin{equation}
 v = \sqrt{\frac{E}{\rho}} 
\end{equation}
gdzie 
\begin{description}
\item [$v$] prędkość rozchodzenia się fali w pręcie $\left[\frac{m}{s}\right]$
\item [$E$]  moduł Younga $\left[\frac{N}{m^2}\right]$
\item [$\rho$] gęstość substancji z której został wykonany pręt $\left[\frac{kg}{m^3}\right]$
\end{description}
czyli
\begin{equation}
\label{wzor:E}
E = v^2\rho 
\end{equation}

\subsection{Długość fali}

\indent W pręcie powstaje fala stojąca, odległość między węzłami fali stojące wynosi $ l = \frac{1}{2} \lambda $. Zależność między  $n$- ta długością fali a długością pręta $L$ wyraża wzór:
\begin{equation}
\label{wzor:lambda}
	\lambda_{n} = \frac{2}{n}L
\end{equation}
gdzie 
\begin{description}
\item [$\lambda_{n}$] długość $n$ -tej fali   [$m$]
\item [$L$] długość pręta [$m$]

\end{description}
Mając długość fali i jej częstotliwość dla $n$ -tej składowej harmonicznej można policzyć prędkość fali $v_{n}$
\begin{equation}
\label{wzor:v}
v_{n} = \lambda_{n} \cdot f_{n}
\end{equation}
gdzie
\begin{description}
\item [$n$] $n$- ta składowa harmoniczna
\item [$v_{n}$] prędkość fali $\left[\frac{m}{s}\right]$
\item [$f_{n}$] częstotliwość dla $n$ -tej składowej [$Hz$]
\end{description}

Pozostawiając do wzoru {\ref{wzor:E}} ostatecznie otrzymujemy:
$$E = 4\rho f^2 l^2$$
 \begin{figure}
 \label{fig:my_label}
 \includegraphics[scale=0.5]{Hooke.jpg} 
 \end{figure}
 
 
\section{Układ pomiarowy}
\begin{itemize}
\item Komputer stacjonarny z oprogramowaniem Zelscope i mikrofonem
\item Zestaw pięciu prętów, o różnych kształtach (stalowe, aluminiowy, miedziany i mosiężny)
\item Suwmiarka, miarka w rolce, waga elektroniczna, młotek
\end{itemize}

\section{Wykonanie ćwiczenia}
 \begin{enumerate}
 \item Ustawienie mikrofonu przy pręcie.
 \item Uderzenie młotkiem w pręt i wciśnięcie stop w programie aby zaobserwować obraz powstały na oscyloskopie.
 \item Zapisanie częstotliwości dla sześciu składowych harmonicznych w tabeli na podstawie obrazu powstałego w programie.
 \item Wyliczenie $n$ długości fali na podstawie wzoru {\ref{wzor:lambda}}, nastepnie $n$ prędkości fali na podstawie wzoru  {\ref{wzor:v}}.
 \item Wyliczenie wartości prędkości średniej 
 $$ v_{\text{śr}} = \frac{\sum^{n}_{i=1}v_{n}}{n}$$
 \item Wyliczenie modułu Younga na podstawie wzoru {\ref{wzor:E}}
 \end{enumerate}
 
 \subsection{Przykładowe obliczenia dla prętu miedzianego}
 \subsubsection{Długość fali $\lambda$}
 \begin{equation}
	\lambda_{n} = \frac{2}{n}L
\end{equation}
 $$ \lambda_{1} = \frac{2}{1}\cdot 1.811 = 3.622 [m]$$
 $$ \lambda_{2} = \frac{2}{2}\cdot 1.811 = 1.811 [m]$$
 $$ \lambda_{3} = \frac{2}{3}\cdot 1.811 = 1.207 [m]$$
 $$ \lambda_{4} = \frac{2}{4}\cdot 1.811 = 0.906 [m]$$
 $$ \lambda_{5} = \frac{2}{5}\cdot 1.811 = 0.724 [m]$$
 $$ \lambda_{6} = \frac{2}{6}\cdot 1.811 = 0.604 [m]$$
 
 \subsubsection{Prędkość fali $v$}
 \begin{equation}
v_{n} = \lambda_{n} \cdot f_{n}
\end{equation}

$$v_{1} = \lambda_{1} \cdot f_{1} = 3721.24 \left[\frac{m}{s}\right]$$

$$v_{2} = \lambda_{2} \cdot f_{2} = 3721.22 \left[\frac{m}{s}\right]$$

$$v_{3} = \lambda_{3} \cdot f_{3} = 3703.67 \left[\frac{m}{s}\right]$$

$$v_{4} = \lambda_{4} \cdot f_{4} = 3737.76 \left[\frac{m}{s}\right]$$

$$v_{5} = \lambda_{5} \cdot f_{5} = 3731.15 \left[\frac{m}{s}\right]$$

$$v_{6} = \lambda_{6} \cdot f_{6} = 3729.71 \left[\frac{m}{s}\right]$$

\subsubsection{Prędkośc średnia fali $v_{\text{śr}}$}
$$ v_{\text{śr}} = \frac{\sum^{n}_{i=1}v_{n}}{n}$$

$$ v_{\text{śr}} = \frac{3721.24 + \cdots + 3729.71}{6} = 3724.13 \left[\frac{m}{s}\right] $$

\subsubsection{Gęstość miedzi}
\begin{equation}
\rho = \frac{m}{v}
\end{equation}

$$\rho = \frac{0.066}{0.375 \cdot \pi \cdot 0.0025^2} = 8963.61 \left[\frac{kg}{m^3}\right]$$
\subsubsection{Moduł Younga $E$}
\begin{equation}
E = v_{\text{śr}}^2\rho 
\end{equation}
$$E = 8963.61 \cdot 3724.13^2 = 124.313 [GPa]$$

\section{Opracowanie wyników pomiarów}
\begin{table}[!htbp]
\resizebox{\textwidth}{!}{%
\begin{tabular}{ll|l|l|}
\hline
\multicolumn{4}{|c|}{\textbf{PRĘT 1 (MIEDŹ)}}                                                                                                                                    \\ \hline
\multicolumn{1}{|l|}{\textbf{Długość l {[}$m${]}}}      & 1.811 & \textbf{Masa  próbki m {[}$kg${]}}  & 0.066 \\ \hline
\multicolumn{1}{|l|}{\textbf{Długość próbki {[}$m${]}}} & 0.375& \textbf{Promień próbki {[}$m${]}}  & 0.0025 \\ \hline
\multicolumn{1}{|l|}{} &                                   & 
\textbf{Gęstość ro $\left[\frac{kg}{m^3}\right]$} & 8963.61                            \\ \hline

\multicolumn{1}{|l|}{\textbf{NR HARMONICZNEJ}}  & \textbf{CZĘSTOTLIWOŚĆ f {[}$Hz${]}} & \textbf{DŁUGOŚĆ FALI $\lambda$ {[}$m${]}}             & \textbf{PRĘDKOŚĆ FALI v $\left[\frac{m}{s}\right]$} \\ \hline
\multicolumn{1}{|l|}{1} & 1024.40 & 3.622 & 3721.24  \\ \hline

\multicolumn{1}{|l|}{2} & 2054.79 & 1.811  & 3721.22  \\ \hline

\multicolumn{1}{|l|}{3} & 3068.49 & 1.207  & 3703.67  \\ \hline

\multicolumn{1}{|l|}{4} & 4123.29 & 0.907  & 3737.76  \\ \hline

\multicolumn{1}{|l|}{5} & 5150.68 & 0.724  & 3731.15  \\ \hline

\multicolumn{1}{|l|}{6} & 6178.08 & 0.604  & 3729.71  \\ \hline

&   & \textbf{ŚREDNIA PRĘDKOŚĆ v $\left[\frac{m}{s}\right]$}  & 3724.13   \\ \cline{3-4} 
&   & \textbf{MODUŁ YOUNGA {[}$GPa${]}}                & 124.313    \\ \cline{3-4} 
\end{tabular}%
}
\end{table}
\begin{table}[!htbp]
\resizebox{\textwidth}{!}{%
\begin{tabular}{ll|l|l|}
\hline
\multicolumn{4}{|c|}{\textbf{PRĘT 2 (STAL) przekój prostokatny}}                                                                                                                                    \\ \hline
\multicolumn{1}{|l|}{\textbf{Długość l {[}$m${]}}}     & 1.802 & \textbf{Masa  próbki m {[}$kg${]}}  & 0.031 \\ \hline
\multicolumn{1}{|l|}{\textbf{Długość próbki {[}$m${]}}} & 0.020  & \textbf{Szerokość próbki {[}$m${]}}  & 0.014 \\ \hline
\multicolumn{1}{|l|}{\textbf{Wysokość próbki {[}$m${]}}} & 0.015 & \textbf{Gęstość ro $\left[\frac{kg}{m^3}\right]$} & 7635.47                            \\ \hline

\multicolumn{1}{|l|}{\textbf{NR HARMONICZNEJ}}  & \textbf{CZĘSTOTLIWOŚĆ f {[}$Hz${]}} & \textbf{DŁUGOŚĆ FALI $\lambda$ {[}$m${]}}             & \textbf{PRĘDKOŚĆ FALI v $\left[\frac{m}{s}\right]$} \\ \hline
\multicolumn{1}{|l|}{1} & 1397.26 & 2.604 & 3632.88  \\ \hline

\multicolumn{1}{|l|}{2} & 2917.81 & 1.802  & 5257.89  \\ \hline

\multicolumn{1}{|l|}{3} & 4301.37 & 1.201  & 5165.95  \\ \hline

\multicolumn{1}{|l|}{4} & 5698.63 & 0.901  & 5134.47  \\ \hline

\multicolumn{1}{|l|}{5} & 7136.99 & 0.708  & 5144.34  \\ \hline

\multicolumn{1}{|l|}{6} & 8616.44 & 0.601  & 5175.90  \\ \hline

&   & \textbf{ŚREDNIA PRĘDKOŚĆ v $\left[\frac{m}{s}\right]$}  & 5175.71   \\ \cline{3-4} 
&   & \textbf{MODUŁ YOUNGA {[}$GPa${]}}                & 204,538    \\ \cline{3-4} 
\end{tabular}%
}
\end{table}
\begin{table}[!htbp]
\resizebox{\textwidth}{!}{%
\begin{tabular}{ll|l|l|}
\hline
\multicolumn{4}{|c|}{\textbf{PRĘT 3 (ALUMINIUM)}}                                                                                                                                    \\ \hline
\multicolumn{1}{|l|}{\textbf{Długość l {[}$m${]}}}      & 0.996 & \textbf{Masa  próbki m {[}$kg${]}}  & 0.024 \\ \hline
\multicolumn{1}{|l|}{\textbf{Długość próbki {[}$m${]}}} & 0.440& \textbf{Promień próbki {[}$m${]}}  & 0.0025 \\ \hline
\multicolumn{1}{|l|}{} &                                   & 
\textbf{Gęstość ro $\left[\frac{kg}{m^3}\right]$} & 2777.98                            \\ \hline

\multicolumn{1}{|l|}{\textbf{NR HARMONICZNEJ}}  & \textbf{CZĘSTOTLIWOŚĆ f {[}$Hz${]}} & \textbf{DŁUGOŚĆ FALI $\lambda$ {[}$m${]}}             & \textbf{PRĘDKOŚĆ FALI v $\left[\frac{m}{s}\right]$} \\ \hline
\multicolumn{1}{|l|}{1} & 2424.66 & 1.992 & 4829.92  \\ \hline

\multicolumn{1}{|l|}{2} & 4972.60 & 0.996  & 4952.71  \\ \hline

\multicolumn{1}{|l|}{3} & 7383.56 & 0.664  & 4902.68  \\ \hline

\multicolumn{1}{|l|}{4} & 9794.52 & 0.498  & 4877.67  \\ \hline

\multicolumn{1}{|l|}{5} & 12294.5 & 0.398  & 4898.14  \\ \hline

\multicolumn{1}{|l|}{6} & 14828.8 & 0.332  & 4923.15  \\ \hline

&   & \textbf{ŚREDNIA PRĘDKOŚĆ v $\left[\frac{m}{s}\right]$}  & 4897.38   \\ \cline{3-4} 
&   & \textbf{MODUŁ YOUNGA {[}$GPa${]}}                &  66.6279    \\ \cline{3-4} 
\end{tabular}%
}
\end{table}
\begin{table}[!htbp]
\resizebox{\textwidth}{!}{%
\begin{tabular}{ll|l|l|}
\hline
\multicolumn{4}{|c|}{\textbf{PRĘT 4 (MOSIĄDZ)}}                                                                                                                                    \\ \hline
\multicolumn{1}{|l|}{\textbf{Długość l {[}$m${]}}}     & 0.995 & \textbf{Masa  próbki m {[}$kg${]}}  & 0.173 \\ \hline
\multicolumn{1}{|l|}{\textbf{Długość próbki {[}$m${]}}} & 0.221  & \textbf{Szerokość próbki {[}$m${]}}  & 0.0095 \\ \hline
\multicolumn{1}{|l|}{\textbf{Wysokość próbki {[}$m${]}}} & 0.010 & \textbf{Gęstość ro $\left[\frac{kg}{m^3}\right]$} & 8240.06                            \\ \hline

\multicolumn{1}{|l|}{\textbf{NR HARMONICZNEJ}}  & \textbf{CZĘSTOTLIWOŚĆ f {[}$Hz${]}} & \textbf{DŁUGOŚĆ FALI $\lambda$ {[}$m${]}}             & \textbf{PRĘDKOŚĆ FALI v $\left[\frac{m}{s}\right]$} \\ \hline
\multicolumn{1}{|l|}{1} & 1684.93 & 1.99 & 3353.01  \\ \hline

\multicolumn{1}{|l|}{2} & 3479.45 & 0.995  & 3462.05  \\ \hline

\multicolumn{1}{|l|}{3} & 5123.29 & 0.663  & 3398.28  \\ \hline

\multicolumn{1}{|l|}{4} & 6917.81 & 0.498 & 3441.61 \\ \hline

\multicolumn{1}{|l|}{5} & 8630.14 & 0.398  & 3434.80  \\ \hline

\multicolumn{1}{|l|}{6} & 10342.5 & 0.332  & 3430.60  \\ \hline

&   & \textbf{ŚREDNIA PRĘDKOŚĆ v $\left[\frac{m}{s}\right]$}  & 3420.06   \\ \cline{3-4} 
&   & \textbf{MODUŁ YOUNGA {[}$GPa${]}}                & 96.3841    \\ \cline{3-4} 
\end{tabular}%
}
\end{table}

\begin{table}[!htbp]
\resizebox{\textwidth}{!}{%
\begin{tabular}{ll|l|l|}
\hline
\multicolumn{4}{|c|}{\textbf{PRĘT 5 (STAL) przekrój okrągły}}                                                                                                                                    \\ \hline
\multicolumn{1}{|l|}{\textbf{Długość l {[}$m${]}}}     & 1.8 & \textbf{Masa  próbki m {[}$kg${]}}  & 0.066 \\ \hline
\multicolumn{1}{|l|}{\textbf{Długość próbki {[}$m${]}}} & 0.020  & \textbf{Szerokość próbki {[}$m${]}}  & 0.0014 \\ \hline
\multicolumn{1}{|l|}{\textbf{Wysokość próbki {[}$m${]}}} & 0.015 & \textbf{Gęstość ro $\left[\frac{kg}{m^3}\right]$} & 7635.47                            \\ \hline

\multicolumn{1}{|l|}{\textbf{NR HARMONICZNEJ}}  & \textbf{CZĘSTOTLIWOŚĆ f {[}$Hz${]}} & \textbf{DŁUGOŚĆ FALI $\lambda$ {[}$m${]}}             & \textbf{PRĘDKOŚĆ FALI v $\left[\frac{m}{s}\right]$} \\ \hline
\multicolumn{1}{|l|}{1} & 1383.56 & 3.60 & 4980.82  \\ \hline

\multicolumn{1}{|l|}{2} & 2917.81 & 1.80  & 5252.06  \\ \hline

\multicolumn{1}{|l|}{3} & 4273.97 & 1.20  & 5128.76  \\ \hline

\multicolumn{1}{|l|}{4} & 5739.73 & 0.90  & 5165.76  \\ \hline

\multicolumn{1}{|l|}{5} & 7219.18 & 0.72  & 5197.81  \\ \hline

\multicolumn{1}{|l|}{6} & 8602.74 & 0.60  & 5161.64  \\ \hline

&   & \textbf{ŚREDNIA PRĘDKOŚĆ v $\left[\frac{m}{s}\right]$}  & 4897.34   \\ \cline{3-4} 
&   & \textbf{MODUŁ YOUNGA {[}$GPa${]}}                & 202,339    \\ \cline{3-4} 
\end{tabular}%
}
\end{table}
\newpage
\section{Opracowanie wyników}
	Dla obliczeń błędów pomiaru przyjęto następujące niepewności:\\
	Dla długości pręta: $u(l)= 1 [mm]$ z powodu niepewności miarki w rolce równej 1 $mm$ \\
	Dla promienia: $u(r)=u(R)= 0,1[mm]$ ponieważ niepewność suwmiarki to 0,01 $mm$\\
	Dla masy próbki:$u(m)= 1 [g]$ niepewność pomiaru wagi przyjmujemy na 1 $g$\\
	Dla częstotliwości:$u(f)= 25 [Hz]$ z powodów tj. niedokładne odczytanie częstotliwości z obrazu oscyloskopu, zakłócenia spowodowane ograniczeniami sprzętu \\
	
\subsection{Niepewność gęstości}
\subsubsection{Niepewność gęstości próbki o kształcie walca}  
	$$ u(\rho)_{w}=\sqrt{\bigg(\frac{\partial \rho}{\partial m}u(m)\bigg)^2+\bigg(\frac{\partial \rho}{\partial l}u(l)\bigg)^2+\bigg(\frac{\partial \rho}{\partial r}u(r)\bigg)^2+} = $$ 
   $$ = \sqrt{\bigg(\frac{1}{\Pi r^2 l)}u(m)\bigg)^2+\bigg(\frac{-m}{l^2 \Pi(r^2)}u(l)\bigg)^2+\bigg(\frac{-2m}{\Pi r^3 l}u(r)\bigg)^2}$$
	gdzie 
    \begin{description}
    \item [$\rho$] gęstość materiału z jakiego został wykonany pręt $\left[\frac{kg}{m^3}\right]$
    
    \item [$r(\rho)$] niepewność gęstości materiału z jakiego został wykonany pręt $\left[\frac{kg}{m^3}\right]$
  
    \item [$u(m)$] niepewność pomiaru masy próbki pręta [$kg$]
  
     \item [$u(l)$] niepewność pomiaru długości próbki pręta
 
     \item [$u(r)$] niepewność promienia próbki

     \item [$u(R)$] niepewność pomiaru wysokości próbki próbki
    \end{description}
    
	\subsection{Niepewność długości fali}
	$$ u(\lambda)=\sqrt{\bigg(\frac{2}{n}u(l)\bigg)^2}$$
	gdzie
    \begin{description}
    \item [$u(\lambda)$] niepewność pomiaru długości fali [$m$]
    \item [$n$] liczba harmonicznych
    \item [$u(l)$] niepewność długości pręta [$m$]
    \end{description}
    
	\subsection{Niepewność prędkości fali}
	$$ u(v)=\sqrt{\bigg(\frac{\partial v}{\partial f}u(f)\bigg)^2+\bigg(\frac{\partial v}{\partial \lambda}u(\lambda)\bigg)^2}=\sqrt{\bigg(\lambda u(f)\bigg)^2+\bigg(f u(\lambda)\bigg)^2}$$
	gdzie 
    \begin{description}
    \item [$u(v)$] niepewność pomiaru prędkości fali $\left[\frac{m}{s}\right]$
    \item [$u(f)$] niepewność pomiaru częstotliwości fali [$Hz$]
    \item [$f$] częstotliwość fali [$Hz$]
     \item [$u(\lambda)$] niepewność pomiaru długości fali [$m$]
     \item [$\lambda$] długość fali [$m$]
    \end{description}
    
	\subsection{Niepewność modułu Younga}
	$$ u(E)=\sqrt{\bigg(\frac{\partial E}{\partial \rho}u(\rho)\bigg)^2+\bigg(\frac{\partial E}{\partial v}u(v)\bigg)^2} =
	\sqrt{\bigg(v^2 u(\rho)\bigg)^2+\bigg(2 \rho v u(v)\bigg)^2}$$
	gdzie
    \begin{description}
    \item [$u(E)$] Niepewność wyznaczonego modułu Younga [$GPa$]
    
    \item [$v$] $\left[\frac{m}{s}\right]$
    
    \item [$u(\rho)$] $\left[\frac{kg}{m^3}\right]$
    
    \item [$\rho$] $\left[\frac{kg}{m^3}\right]$
    \item [$u(v)$] $\left[\frac{m}{s}\right]$
    \end{description}
    
\begin{table}[!htbp]
\resizebox{\textwidth}{!}{%
\begin{tabular}{|l|l|l|l|}
\hline
\multicolumn{1}{|c|}{\textbf{Nr pręta (materiał)}} & \multicolumn{1}{c|}{\textbf{Niepewność gęstości u($\rho$) $\left[\frac{kg}{m^3}\right]$}} & \multicolumn{1}{c|}{\textbf{Niepewność prędkości fali u(v) $\left[\frac{m}{s}\right]$}} & \multicolumn{1}{c|}{\textbf{Niepewność moduług Younga u(E) {[}$GPa${]}}} \\ \hline
1 (aluminium)                                      & 35.4413                                                                            & 90.0115                                                                & 5.9761                                                                 \\ \hline
2 (stal)                                           & 571.8466                                                                           & 90.0159                                                                & 16.6563                                                                \\ \hline
3 (mosiądz)                                        & 80.7742                                                                            & 50.0353                                                                & 3.1301                                                                 \\ \hline
4 (stal)                                           & 286.3508                                                                           & 90.0157                                                                & 10.2482                                                                \\ \hline
5 (stal)                                           & 78.9227                                                                            & 90.0159                                                                & 7.1581                                                                 \\ \hline
\end{tabular}%
}
\end{table}
\section{Wnioski}
\begin{itemize}
\item 
\item 
\item  
\end{itemize}
\end{document}